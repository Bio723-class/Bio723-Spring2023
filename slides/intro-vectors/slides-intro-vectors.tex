%!TEX program = xelatex
\documentclass{beamer}

%\useoutertheme[glossy]{wuerzburg}
\useinnertheme[shadow,outline]{chamfered}
%\usecolortheme{shark}
\usecolortheme{beaver}
\beamertemplatenavigationsymbolsempty


\usepackage{fontspec}      % support opentype text fonts
\usepackage{amsmath}
\usepackage[mathrm=sym,mathbf=sym]{unicode-math}
                           % support opentype math fonts, including bold

% Slightly reduce font size.
\defaultfontfeatures{Scale=.85}
%
% The explicit .otf extension tells the engines to find the font via
% filenames in the TeX trees, rather than as system-installed fonts.
%
\setmainfont[% main rm
  Ligatures=TeX,
      ItalicFont=LucidaBrightOT-Italic.otf,
        BoldFont=LucidaBrightOT-Demi.otf,
  BoldItalicFont=LucidaBrightOT-DemiItalic.otf,
  ]{LucidaBrightOT.otf}
%
\setsansfont[% main sans
  Ligatures=TeX,
      ItalicFont=LucidaSansOT-Italic.otf,
        BoldFont=LucidaSansOT-Demi.otf,
  BoldItalicFont=LucidaSansOT-DemiItalic.otf,
  ]{LucidaSansOT.otf}
%
\setmonofont[% main typewriter
      ItalicFont=LucidaSansTypewriterOT-Oblique.otf,
        BoldFont=LucidaSansTypewriterOT-Bold.otf,
  BoldItalicFont=LucidaSansTypewriterOT-BoldOblique.otf,
  ]{LucidaSansTypewriterOT.otf}
%

\setmathfont{LucidaBrightMathOT.otf}[BoldFont=LucidaBrightMathOT-Demi]
%
% The specialized one-off fonts:
\newfontface\LucidaBlackletter{LucidaBlackletterOT.otf}
\newfontface\LucidaCalligraphy{LucidaCalligraphyOT.otf}
\newfontface\LucidaHandwriting{LucidaHandwritingOT.otf}
%
% GrandeMono and Console fonts for an example:
\newfontface\LucidaGrandeMono{LucidaGrandeMonoDK.otf}
\newfontface\LucidaConsole{LucidaConsoleDK.otf}

\usepackage{fancyvrb}

%% Fancy syntax coloring via pygments
%\usepackage{minted}
%\definecolor{bg}{rgb}{0.95,0.95,0.95}
%\usemintedstyle{borland}


% \newenvironment{Rcode}
% {\VerbatimEnvironment
%  \begin{minted}[fontsize=\scriptsize,baselinestretch=1]{r}}%
% {\end{minted}}

% \newenvironment{Pcode}
% {\VerbatimEnvironment
% \begin{minted}[fontsize=\scriptsize,baselinestretch=1]{python}}%
% {\end{minted}}

% \newenvironment{Code}[1]
% {\VerbatimEnvironment
 % \begin{minted}[fontsize=\scriptsize,baselinestretch=1]{#1}}%
% {\end{minted}}


\usepackage{textfit} % commands \scaletoheight{height}{text} and \scaletowidth{width}{text}

\usepackage{tikz}

\usepackage{tcolorbox}

\usepackage{subcaption}

\newtheorem{Alert}{Alert}
\newtheorem{Highlight}{Highlight}

\newcommand{\Species}[1]{{\rmfamily \itshape #1}}
\newcommand{\Real}{\ensuremath{\mathbb{R}}}
\newcommand{\RealN}{\ensuremath{\mathbb{R}^n}}
\newcommand{\RealP}{\ensuremath{\mathbb{R}^p}}
\newcommand{\Mtx}[1]{\ensuremath{\bm{#1}}}
\newcommand{\Inv}[1]{\ensuremath{#1^{-1}}}
\newcommand{\InvMtx}[1]{\ensuremath{\bm{#1}^{-1}}}
\newcommand{\Red}[1]{\textcolor{red}{#1}}
\newcommand{\PsInv}[1]{\ensuremath{\bm{#1}^{+}}}
\DeclareMathOperator{\cov}{cov}
\DeclareMathOperator{\corr}{corr}

\usepackage{booktabs}


\usepackage{pifont}
\newcommand{\weblink}{\ding{43}}  % hand with pointing finger

\definecolor{links}{HTML}{2A1B81}
\hypersetup{colorlinks,linkcolor=,urlcolor=magenta}




% --- Override \vec with an invocation of \xvec.
\let\stdvec\vec
\newcommand{\bvec}[1]{\vec{\mathbf{#1}}}



%===========================================================
% Title Info
\title{Scientific Computing for Biologists}
\subtitle{Data as Vectors: Introduction to Vector Geometry} % (optional)
\date{}
\author[P. Magwene]{Instructor: Paul M. Magwene}


\begin{document}
%===========================================================
\begin{frame}
\titlepage
\end{frame}

%===========================================================
\begin{frame}
  \frametitle{Overview of Lecture}

\begin{itemize}
        \item Variable space/Subject space representations
		\item Vector Geometry
		\begin{itemize}
			\item Vectors are directed line segments
			\item Vector length
		\end{itemize}
		\item Vector Arithmetic
		\begin{itemize}
			\item Addition, subtraction
			\item Scalar multiplication
			\item Linear combinations of vectors
			\item Dot product and projection
		\end{itemize}
		\item Vector representations of multivariate data
        \begin{itemize}
			\item Mean as projection in subject space
			\item Bivariate regression in geometric terms
%			\item Difference in group means as a regression problem
		\end{itemize}
\end{itemize}

\end{frame}
%===========================================================

% %===========================================================
% \begin{frame}
%   \frametitle{Hands-on Session}
% 		\begin{itemize}
%           \item Visualizing bivariate relationships in R
% 		  \item Vector mathematics in R
% 		  \item Writing functions in R
% 		  \item Correlation and linear regression in R
% 	 \end{itemize}
% \end{frame}
% %===========================================================

%===========================================================
\begin{frame}
  \frametitle{Variable Space Representation of a Data Set}

Consider a data set in which we've measured variables $\mathbf{X} = {X_1, X_2,\ldots,X_p}$, on a set of subjects (objects) $a_1,...,a_n$.

\begin{center}
    \begin{tabular}{l|cc}

 & $X_1$ & $X_2$ \\ \hline

$a_1$ & 0.9 & 1.4 \\
$a_2$ & 1.1 & 1.7 \\
$\vdots$ & $\vdots$ & $\vdots$ \\
$a_n$ & 0.5 &  1.55

    \end{tabular}
\end{center}

Such data is most often represented by drawing the objects as points in space of dimension $p$. This is the \emph{variable space representation} of the data.

\begin{center}

\begin{tikzpicture}[x=0.25cm, y=0.25cm]

\draw[->] (-8,0) -- (8,0);
\draw (8,0) node[right,font=\footnotesize] {$X_1$};

\draw[->] (0,-4) -- (0,4);
\draw (0,4) node[above,font=\footnotesize] {$X_2$};

\draw plot[only marks, mark=ball] coordinates { (-3.5,-1) (-3,-2) (-1,-1.5) (-1,0.25) (0,-1.5) (0,0.1) (0.5,1.6) (1,2.5) (2,0.5) (2,3) (3,2.5) };

\end{tikzpicture}
\end{center}


\end{frame}
%===========================================================

%===========================================================
\begin{frame}
  \frametitle{Subject Space Representation of a Data Set}

\begin{small}

An alternate representation is to consider the variables in the space of the subjects. This is the \emph{subject space} representation.
% \begin{itemize}
% \item trickier to visualize because high dimensional
% \end{itemize}

\medskip

How do we come up with a useful representation of variables in subject space?
\begin{itemize}
 % \item Any pair of non-parallel vectors (of arbitrary dimension) defines a plane.
 \item Let the variables be represented by centered vectors
 \begin{itemize}
    \item lengths of vectors are proportional to standard deviation
    \item angle between vectors represents association or similarity
 \end{itemize}
\end{itemize}
\end{small}

\begin{center}

\begin{tikzpicture}[x=0.5cm, y=0.5cm]

\draw[thick,->] (0,0) -- (3,2);
\draw (3,2) node[right] {$\bvec{x_1}$};

\draw[thick,->] (0,0) -- (5,-1);
\draw (5,-1) node[above] {$\bvec{x_2}$};

\end{tikzpicture}
\end{center}

This representation of variables as vectors in the space of the subjects is the view that we'll develop over the next few lectures.


\end{frame}
%===========================================================




%===========================================================
\begin{frame}
  \frametitle{Vector Geometry}

Vectors are directed line segments.

\[
\bvec{x} = \left[ 
\begin{array}{c} 
  x_1 \\
  x_2 \\
  \vdots \\
  x_n \\
\end{array} \right] = [x_1,x_2, \cdots, x_n]'
\]

\begin{center}

\begin{tikzpicture}[x=0.35cm, y=0.35cm]
\draw[step=1.0, style=help lines] (-5.9,-2.9) grid (5.9,5.9);

\draw (-6,0) -- (6,0);
\draw (0,-3) -- (0,5);

\draw[thick,red,->] (0,0) -- (3,4);
\draw (3,4) node[above right] {$\bvec{x}$};

\draw[thick,blue,->] (-1,-1) -- (-2,5);
\draw (-2,5) node[above right] {$\bvec{y}$};

\end{tikzpicture}

\end{center}

All of the figures and algebraic formulas I show you apply to $n$-dimensional vectors.


\end{frame}
%===========================================================

%===========================================================
\begin{frame}
  \frametitle{Vector Geometry}

Vectors have direction and length:
 
\[
\bvec{x} = [x_1,x_2]' = [2,3]';\; |\bvec{x}| = \sqrt{x_1^2 + x_2^2 + \cdots + x_n^2}
\]

\begin{center}

\begin{tikzpicture}[x=0.4cm, y=0.4cm]


\draw [step=1.0,style=help lines] (-5.9,-4.9) grid (5.9,4.9);
\draw (-6,0) -- (6,0);
\draw (0,-5) -- (0,5);

\draw[thick,red,->] (0,0) -- (2,3);
\draw (2,3) node[above right] {$\bvec{x}$};

\draw[thick,blue,->] (1,-3) -- (3,0);
\draw (3,0) node[above right] {$\bvec{y}$};

\end{tikzpicture}
\end{center}

Often starting point is ignored, in which case $\bvec{x} = \bvec{y}$.


\end{frame}
%===========================================================

%===========================================================
\begin{frame}
  \frametitle{Scalar Multiplication of a Vector}

Let $k$ be a scalar.

\[
k \bvec{x} = \left[ 
\begin{array}{c} k x_1 \\
                 k x_2 \\
                 \vdots \\
                 k x_n \\
\end{array} \right]
\]

\begin{center}

\begin{tikzpicture}[x=0.5cm, y=0.5cm]

\draw[step=0.5cm, style=help lines] (-1.9,-1.9) grid (6.9,4.9);
\draw (-2,0) -- (6,0);
\draw (0,-2) -- (0,5);

\draw[thick,red,->] (0,0) -- (2,1);
\draw (2,1) node[below right] {$\bvec{x}$};

\draw[thick,->] (2,1) -- (6,3);
\draw (6,3) node[above left] {$3 \bvec{x}$};

\end{tikzpicture}

$\bvec{x} = [2,1]'; \; 3\bvec{x} = [6,3]'$.

\end{center}




\end{frame}
%===========================================================

%===========================================================
\begin{frame}
  \frametitle{Vector Addition}

Let $\bvec{x} = [2,1]'; \; \bvec{y} = [1,3]'$

\[
\bvec{z} = \bvec{x} + \bvec{y} = \left[ \begin{array}{c} x_1 + y_1 \\
															x_2 + y_2 \\
															\vdots \\
															x_n + y_n
\end{array} \right]
\]

\begin{center}

\begin{tikzpicture}[x=0.5cm, y=0.5cm]

\draw[step=0.5cm, style=help lines] (-1.9,-1.9) grid (5.9,4.9);
\draw (-2,0) -- (6,0);
\draw (0,-2) -- (0,5);

\draw[thick,red,->] (0,0) -- (2,1);
\draw (2,1) node[below right] {$\bvec{x}$};

\draw[thin,blue,dashed,->] (2,1) -- ++(1,3);

\draw[thick,blue,->] (0,0) -- (1,3);
\draw (1,3) node[above left] {$\bvec{y}$};


\draw[thick,->] (0,0) -- (3,4);
\draw (3,4) node[above right] {$\bvec{z}$};

\end{tikzpicture}
\end{center}

Addition follows the `head-to-tail' rule.


\end{frame}
%===========================================================

%===========================================================
\begin{frame}
  \frametitle{Vector Subtraction}

Let $\bvec{x} = [2,1]'; \; \bvec{y} = [1,3]'$

\[
\bvec{z} = \bvec{x} - \bvec{y} = \left[ \begin{array}{c} x_1 - y_1 \\ x_2 - y_2 \end{array} \right] =
															\left[ \begin{array}{c} 1 \\ -2 \end{array} \right]
\]

\begin{center}

\begin{tikzpicture}[x=0.5cm, y=0.5cm]

\draw[step=0.5cm, style=help lines] (-1.9,-2.9) grid (5.9,4.9);
\draw (-2,0) -- (6,0);
\draw (0,-3) -- (0,4);

\draw[thick,red,->] (0,0) -- (2,1);
\draw (2,1) node[below right] {$\bvec{x}$};

\draw[thin,blue,dashed,->] (2,1) -- ++(-1,-3);

\draw[thick,blue,->] (0,0) -- (1,3);
\draw (1,3) node[above left] {$\bvec{y}$};


\draw[thick,->] (0,0) -- (1,-2);
\draw (1,-2) node[above right] {$\bvec{z}$};

\end{tikzpicture}
\end{center}

Follow the addition rule for $-1 \bvec{y}$.


\end{frame}
%===========================================================

%===========================================================
\begin{frame}
  \frametitle{Linear Combinations of Vectors}

A linear combination of vectors is of the form $z = b_1 \bvec{x} + b_2 \bvec{y}$


\[
\bvec{z} = 3\bvec{x} - 0.5\bvec{y} = 3 \left[ \begin{array}{c} x_1  \\ x_2  \end{array} \right] -
															0.5 \left[ \begin{array}{c} y_1 \\ y_2 \end{array} \right]
\]

\begin{center}

\begin{tikzpicture}[x=0.5cm, y=0.5cm]

\draw[step=0.5cm, style=help lines] (-1.9,-2.9) grid (6.9,4.9);
\draw (-2,0) -- (7,0);
\draw (0,-3) -- (0,4);

\draw[thick,red,->] (0,0) -- (2,1);
\draw (2,1) node[above left] {$\bvec{x}$};

\draw[thin,red,dashed,->] (2,1) -- (6,3);
\draw (6,3) node[above left,font=\footnotesize] {$3\bvec{x}$};

\draw[thin,blue,dashed,->] (6,3) -- (5.5,1.5);
\draw (5.5, 1.5) node[above right, font=\footnotesize] {$-0.5\bvec{y}$};

\draw[thick,blue,->] (0,0) -- (1,3);
\draw (1,3) node[above left] {$\bvec{y}$};

\draw[thick,->] (0,0) -- (5.5, 1.5);
\draw (5.5, 1.5) node[below] {$\bvec{z}$};

\end{tikzpicture}
\end{center}


\end{frame}
%===========================================================

%===========================================================
\begin{frame}
  \frametitle{Dot Product}

The dot (inner) product of two vectors, $\bvec{x} \cdot \bvec{y}$ is a scalar.
%
\begin{eqnarray*}
\bvec{x} \cdot \bvec{y}  & = & x_1 y_1 + x_2 y_2 + \cdots + x_n y_n \\
					   & = & |\bvec{x}||\bvec{y}| \cos \theta
\end{eqnarray*}
%
where $\theta$ is the angle (in radians) between $\bvec{x}$ and $\bvec{y}$

\begin{center}

\begin{tikzpicture}[x=0.5cm, y=0.5cm]

\draw[step=0.5cm, style=help lines] (-1.9,-2.9) grid (6.9,4.9);
\draw (-2,0) -- (7,0);
\draw (0,-3) -- (0,4);

\draw[thick,red,->] (0,0) -- (3,2);
\draw (2,1) node[below right] {$\bvec{x}$};

\draw[thick,blue,->] (0,0) -- (1,3);
\draw (1,3) node[above left] {$\bvec{y}$};

\draw (1.5,1) arc (34:69:1cm);
\path (0,0) ++(55:1.1cm) node[font=\footnotesize] {$\theta$};

\end{tikzpicture}

$ \bvec{x} = [3,2]', \bvec{y} = [1,3]'; \; \bvec{x} \cdot \bvec{y} = \sqrt{13}\sqrt{10}\cos \theta = 9$

\end{center}


\end{frame}
%===========================================================

%===========================================================
\begin{frame}
  \frametitle{Useful Geometric Quantities as Dot Product}


Length:
\begin{eqnarray*}
|\bvec{x}|^2 & = & \bvec{x} \cdot \bvec{x}  =  x_1^2 + x_2^2 + \cdots + x_n^2\\
|\bvec{y}|^2 & = & \bvec{y} \cdot \bvec{y}
\end{eqnarray*}

Distance:
\begin{eqnarray*}
|\bvec{x}-\bvec{y}|^2 = \bvec{x} \cdot \bvec{x} + \bvec{y} \cdot \bvec{y} - 2 \bvec{x} \cdot \bvec{y}
\end{eqnarray*}

Angle:
\begin{eqnarray*}
\cos \theta = \frac{\bvec{x} \cdot \bvec{y}}{|x||y|}
\end{eqnarray*}

\begin{center}

\begin{tikzpicture}[x=0.25cm, y=0.25cm]

\draw (-2,0) -- (7,0);
\draw (0,-3) -- (0,4);

\draw[thick,red,->] (0,0) -- (3,2);
\draw (2,1) node[right,font=\footnotesize] {$\bvec{x}$};

\draw[thick,blue,->] (0,0) -- (1,3);
\draw (1,3) node[above,,font=\footnotesize] {$\bvec{y}$};

\draw (0,0) +(34:0.25cm) arc (34:69:0.25cm);
\path (0,0) ++(55:0.5cm) node[font=\tiny] {$\theta$};

\end{tikzpicture}

\end{center}

\end{frame}
%===========================================================

%===========================================================
\begin{frame}
  \frametitle{Dot Product Properties}

Some additional properties of the dot product that are useful to know:

\begin{eqnarray*}
\bvec{x} \cdot \bvec{y} & = & \bvec{y} \cdot \bvec{x} \ \mbox{(commutative)} \\
\bvec{x} \cdot (\bvec{y} + \bvec{z}) & = & \bvec{x} \cdot \bvec{y} + \bvec{x} \cdot \bvec{z} \ \mbox{(distributive)}\\
(k\bvec{x}) \cdot \bvec{y} & = & \bvec{x} \cdot (k\bvec{y}) = k(\bvec{x} \cdot \bvec{y}) \ \mbox{where $k$ is a scalar} \\
\bvec{x} \cdot \bvec{y} & = & 0 \ \mbox{iff $\bvec{x}$ and $\bvec{y}$ are orthogonal}
\end{eqnarray*}

\end{frame}
%===========================================================

%===========================================================
\begin{frame}
  \frametitle{Useful vectors}

\begin{itemize}
  \item Unit vector in the direction of $\vec{x}$ -- a vector of length one parallel to $\bvec{x}$.  Can be calculated as:
    $$
    \text{Unit vector in the direction of } \bvec{x} = \frac{\bvec{x}}{|\bvec{x}|}
    $$
  
  \item {One-vector} -- a $n$-dimensional vector of where every element is the number 1.
  
  $$
  \bvec{1}_n = \left[
    \begin{array}{c}
    1 \\
    1 \\
    \vdots \\
    1 \\
    \end{array}
  \right]
  $$



  Note that 1-vectors are not, in general, unit vectors!



\end{itemize}

\end{frame}
%===========================================================

%===========================================================
\begin{frame}
  \frametitle{Vector Projection}

The projection of $\bvec{y}$ onto $\bvec{x}$, $P_{\bvec{x}}(\bvec{y})$, is the vector obtained by placing $\bvec{y}$ and $\bvec{x}$ tail to tail and dropping a line, perpendicular to $\bvec{x}$, from the head of $\bvec{y}$ onto the line defined by $\bvec{x}$.
\[
P_{\bvec{x}}(\bvec{y}) = \left(\frac{\bvec{x} \cdot \bvec{y}}{|\bvec{x}|}\right) \frac{\bvec{x}}{|\bvec{x}|} = \left(\frac{\bvec{x} \cdot \bvec{y}}{|\bvec{x}|^2}\right)\bvec{x}
\]

The component of $\bvec{y}$ in $\bvec{x}$, $C_{\bvec{x}}(\bvec{y})$, is the length of $P_{\bvec{x}}(\bvec{y})$.
\[
C_{\bvec{x}}(\bvec{y}) = \frac{\bvec{x} \cdot \bvec{y}}{|\bvec{x}|} = |\bvec{y}|\cos \theta
\]

\begin{center}

\begin{tikzpicture}[x=0.5cm, y=0.5cm]

\draw (-2,0) -- (7,0);
\draw (0,-1.5) -- (0,4);

\draw[thick,red,->] (0,0) -- (3,0);
\draw (3,0) node[below] {$\bvec{x}$};

\draw[thick,blue,->] (0,0) -- (2,3);
\draw (2,3) node[above left] {$\bvec{y}$};

\draw[thin,blue,dashed] (2,3) -- (2,0);

\draw[thick, dashed,->] (0,0) -- (2,0);
\draw (2,0) node[below left,font=\tiny] {$P_{\bvec{x}}(\bvec{y})$};

\draw (0,0) +(0:0.5cm) arc (0:56:0.5cm);
\path (0,0) ++(30:0.6cm) node[font=\tiny] {$\theta$};


\end{tikzpicture}


\end{center}


\end{frame}
%===========================================================

%===========================================================
\begin{frame}
  \frametitle{Vector Projection II}

$\bvec{y}$ can be decomposed into two parts:

1. a vector parallel to $\bvec{x}$, $\widehat{y} = P_{\bvec{x}}(\bvec{y})$, 

\smallskip

2. a vector perpendicular to $\bvec{x}$, $\widehat{y}_{\bot}$.

\[
\bvec{y} = \widehat{y} + \widehat{y}_{\bot}
\]

\begin{center}

\begin{tikzpicture}[x=0.5cm, y=0.5cm]

\draw (-2,0) -- (7,0);
\draw (0,-1.5) -- (0,4);

\draw[thick,red,dashed,->] (0,0) -- (3,0);
\draw (3,0) node[below,font=\small] {$\bvec{x}$};

\draw[thick,blue,->] (0,0) -- (2,3);
\draw (2,3) node[left] {$\bvec{y}$};

\draw[thick, ->] (2,0) -- (2,3);
\draw (2,3) node[below right] {$\widehat{y}_{\bot}$};

\draw[thick,->] (0,0) -- (2,0);
\draw (2,0) node[below left] {$\widehat{y}$};

\end{tikzpicture}
\end{center}

\begin{itemize}
 \item $\widehat{y}$  is the closest vector to $\bvec{y}$ in the subspace defined by $\bvec{x}$, i.e. $|\widehat{y}_{\bot}|$ is as small as possible
 \item $\widehat{y}_{\bot}$ is \emph{orthogonal} to $\widehat{y}$ and $\bvec{x}$.
\end{itemize}

\end{frame}
%===========================================================

%===========================================================
\begin{frame}

\begin{center}
\LARGE{Vector Geometry of Simple Statistics}
\end{center}


\end{frame}
%===========================================================


%===========================================================
\begin{frame}
  \frametitle{Geometry of the Mean in Variables Space}

The mean is a single number summary of a set (vector) of values, $\bvec{x}$.  The mean is `optimal' in that it is the value that minimizes the following quantity:

\begin{equation*}
 \sum_{i=1}^{n}(x_i-\bar{x})^2 
\end{equation*}


\medskip

\begin{center}
  \begin{tikzpicture}[x=0.5cm, y=0.5cm]
  
  \draw[thick,-] (0,0) -- (10,0);
  \draw (10,0) node[right] {$\mathbf{x}$};
  
  \draw plot[only marks, mark=ball] coordinates { (1,0) (4,0) (2,0) (7,0) (3,0) (9,0) (3.5,0) (9.5,0) (8,0) (4.5,0) };
  
  
  \draw[thick,red,->] (5.15,-2) -- (5.15,0);
  \draw (5.15,-2) node[below] {$\bar{x}$};
  
  %\draw[thick,blue,-] (1,1) -- (5.15,1);
  \draw (1,1) edge [color = blue,|-|] node[above] {$x_i - \bar{x}$}  (5.15,1);
  %\draw (3,1.25) node[above] {$x_i - \bar{x}$};
  
  
  \end{tikzpicture}
  \end{center}

% \begin{center}
% \begin{tikzpicture}[x=0.5cm, y=0.5cm]

% \draw[thick,-] (0,0) -- (10,0);
% \draw (10,0) node[right] {$\mathbf{x}$};

% \draw plot[only marks, mark=ball] coordinates { (1,0) (4,0) (2,0) (7,0) (3,0) (9,0) (3.5,0) (9.5,0) (8,0) (4.5,0) };

% \draw[thick,red,->] (5.15,-2) -- (5.15,0);
% \draw (5.15,-2) node[below] {$\bar{x}$};


% \end{tikzpicture}
% \end{center}





\end{frame}
%===========================================================




%===========================================================
\begin{frame}
  \frametitle{Sketch of Proof: Deriving the mean in vector geometric terms, part I}

\begin{itemize}

\item The mean, $\bar{x}$, minimizes the quantity $\sum_{i=1}^{n}(x_i-\bar{x})^2$.


\item The above can be written as $|\vec{\mathbf{x}}-\vec{\mathbf{1}}\bar{x}|^2$ where $\bvec{x} = [x_1, x_2, \ldots, x_n]'$ and $\vec{\mathbf{1}} = [1,1,\ldots,1]'$

\item We are look for the scalar multiple, $\bar{x}$, of the one vector that minimizes $|\vec{\mathbf{x}}-\vec{\mathbf{1}}\bar{x}|^2$

\item What does the geometry of $\bvec{x}$, $\bvec{1}$, and $\vec{\mathbf{x}}-\vec{\mathbf{1}}\bar{x}$ look like?

\end{itemize}

\smallskip

\begin{center}
\begin{tikzpicture}[x=0.5cm, y=0.5cm]

\draw[thick,->] (0,0) -- (6,0);
\draw (6,0) node[above,font=\tiny] {$\vec{\mathbf{1}} $};

\draw[thick,red,->] (0,0) -- (4,0);
\draw (4,0) node[below,font=\footnotesize] {$\bar{x}\bvec{1}$};

\draw[thick,->] (0,0) -- (4,3);
\draw (4,3) node[above] {$\vec{\mathbf{x}}$};

\draw[thick,dashed,blue,->] (4,0) -- (4,3);
\draw (4,2) node[below right,font=\footnotesize] {$\bvec{x} - \bar{x}\bvec{1}$};

\end{tikzpicture}
\end{center}

This picture looks familiar!  Where did we see it before?

\end{frame}
%===========================================================

% %===========================================================
% \begin{frame}
%   \frametitle{Proof: The mean is least-squares optimal, part II}

% \begin{itemize}
% \item Recall that:
% \begin{eqnarray}
% P_{\vec{\mathbf{1}}}(\vec{\mathbf{x}}) & = & \vec{\mathbf{1}}\bar{x} \ \mbox{for some}\ \bar{x} \\
% %
% (\vec{\mathbf{x}} - P_{\vec{\mathbf{1}}}(\vec{\mathbf{x}})) \cdot \vec{\mathbf{1}} & = & 0
% \end{eqnarray}

% \item Substituting (1) into (2):
% \begin{eqnarray}
% (\vec{\mathbf{x}} - \vec{\mathbf{1}}\bar{x}) \cdot \vec{\mathbf{1}} & = & 0 \\
% %
% \vec{\mathbf{x}} \cdot \vec{\mathbf{1}} & = & \bar{x} (\vec{\mathbf{1}} \cdot \vec{\mathbf{1}})
% \end{eqnarray}

% \item Expanding (4):
% \begin{eqnarray}
% x_1 + x_2 + \cdots + x_n & = & n \bar{x} \\
% %
% \sum x_i & = & n \bar{x} \\
% %
% \bar{x} & = & (1/n) \sum x_i
% \end{eqnarray}

% \end{itemize}


% \end{frame}
% %===========================================================




%===========================================================
\begin{frame}
  \frametitle{Sketch of Proof: Deriving the mean in vector geometric terms, part II}

\begin{itemize}

\item The mean can be interpreted in terms of the projection of $\bvec{x}$ onto the 1-vector:

\end{itemize}

\smallskip

\begin{center}
  \begin{tikzpicture}[x=0.5cm, y=0.5cm]
  
  \draw[thick,->] (0,0) -- (6,0);
  \draw (6,0) node[above,font=\tiny] {$\vec{\mathbf{1}} $};
  
  \draw[thick,red,->] (0,0) -- (4,0);
  \draw (4,0) node[below,font=\footnotesize] {$P_{\bvec{\mathbf{1}}}(\bvec{\mathbf{x}}) = \bar{x}\bvec{1}$};
  
  \draw[thick,->] (0,0) -- (4,3);
  \draw (4,3) node[above] {$\vec{\mathbf{x}}$};
  
  \draw[thick,dashed,blue,->] (4,0) -- (4,3);
  \draw (4,2) node[below right,font=\footnotesize] {$\bvec{x} - \bar{x}\bvec{1}$};
  
  \end{tikzpicture}
  \end{center}

  \begin{eqnarray*}
    P_{\bvec{1}}(\bvec{x}) &=& \left(\frac{\bvec{1} \cdot \bvec{x}}{\bvec{1}\cdot\bvec{1}}\right)\bvec{1} \\
                         &=& \bar{x}\bvec{1} \\
                         &=& [\bar{x}, \bar{x}, \ldots, \bar{x}]'   
    \end{eqnarray*}
    


\end{frame}
%===========================================================

%===========================================================
\begin{frame}
  \frametitle{Vector and Algebraic Formulas for the Mean}


Vector formula for the mean:

  \begin{eqnarray*}
   \bar{x} &=& \frac{\bvec{1} \cdot \bvec{x}}{\bvec{1} \cdot \bvec{1}} \\
          %  &=& \frac{\bvec{1} \cdot \bvec{x}}{|\bvec{1}|^2}
  \end{eqnarray*}

  \medskip

Algebraic formula for the mean of $\bvec{x}$:

\begin{equation*}
 \bar{x} = \frac{x_1 + x_2 + \cdots + x_n}{n} = \frac{1}{n}\sum_{i=1}^{n} x_i
\end{equation*}

\medskip




\end{frame}
%===========================================================



% %===========================================================
% \begin{frame}
%   \frametitle{Geometry of the Mean in Subject Space II}

% Geometric derivation of the sample mean:

% \begin{center}
% \begin{tikzpicture}[x=0.5cm, y=0.5cm]

% \draw[thick,->] (0,0) -- (6,0);
% \draw (6,0) node[above,font=\tiny] {$\bvec{\mathbf{1}} $};

% \draw[thick,red,->] (0,0) -- (4,0);
% \draw (4,0) node[below,font=\footnotesize] {$P_{\bvec{\mathbf{1}}}(\bvec{\mathbf{x}}) = [\bar{x}, \bar{x},\ldots,\bar{x}]$};

% \draw[thick,->] (0,0) -- (4,3);
% \draw (4,3) node[above] {$\bvec{\mathbf{x}}$};

% \draw[dashed,blue,->] (4,0) -- (4,3);
% \draw (4,3) node[below right,font=\footnotesize] {$\bvec{e} = \bvec{\mathbf{x}} - P_{\bvec{\mathbf{1}}}(\bvec{\mathbf{x}}) = [x_1-\bar{x}, x_2 - \bar{x},\ldots,x_n - \bar{x}]$};

% \end{tikzpicture}
% \end{center}


% \end{frame}
% %===========================================================

%===========================================================
\begin{frame}
\frametitle{Variable Space Geometry of Sample Variance}

Sample variance is proportional to the sum of squared deviates about the mean:

\[
S_x^2 = \frac{1}{(n-1)} \sum (x_i - \bar{x})^2
\]

\medskip



\begin{center}
\begin{tikzpicture}[x=0.5cm, y=0.5cm]

\draw[thick,-] (0,0) -- (10,0);
\draw (10,0) node[right] {$\mathbf{x}$};

\draw plot[only marks, mark=ball] coordinates { (1,0) (4,0) (2,0) (7,0) (3,0) (9,0) (3.5,0) (9.5,0) (8,0) (4.5,0) };


\draw[thick,red,->] (5.15,-2) -- (5.15,0);
\draw (5.15,-2) node[below] {$\bar{x}$};

%\draw[thick,blue,-] (1,1) -- (5.15,1);
\draw (1,1) edge [color = blue,|-|] node[above] {$x_i - \bar{x}$}  (5.15,1);
%\draw (3,1.25) node[above] {$x_i - \bar{x}$};


\end{tikzpicture}
\end{center}



\end{frame}
%===========================================================



%===========================================================
\begin{frame}[shrink]
  \frametitle{Vector Geometry of Sample Variance}

\begin{itemize}

\item Let $\bvec{e_x} = \bvec{x} - \bar{x}\bvec{1}$


\medskip

\begin{center}

  \begin{tikzpicture}[x=0.5cm, y=0.5cm]
  
    \draw[thick,->] (0,0) -- (6,0);
    \draw (6,0) node[above,font=\tiny] {$\vec{\mathbf{1}} $};
    
    \draw[thick,red,->] (0,0) -- (4,0);
    \draw (4,0) node[below,font=\footnotesize] {$\bar{x}\bvec{1}$};
    
    \draw[thick,->] (0,0) -- (4,3);
    \draw (4,3) node[above] {$\vec{\mathbf{x}}$};
    
    \draw[thick,dashed,blue,->] (4,0) -- (4,3);
    \draw (4,2) node[below right,font=\footnotesize] {$\bvec{e} = \bvec{x} - \bar{x}\bvec{1}$};
    
    \end{tikzpicture}  

  \end{center}

\end{itemize}

\medskip

The sample variance can be expressed in terms of dots products of $\bvec{e_x}$ with itself:

\begin{equation*}
    S_{x}^2 = \frac{\bvec{e}_x \cdot \bvec{e}_x}{n-1} = \frac{|\bvec{e}_x|^2}{n-1}
\end{equation*}




\end{frame}
%===========================================================

%===========================================================
\begin{frame}
  \frametitle{Mean centering}

In the previous slide, we considered the vector: 

\[
\bvec{e_x} = \bvec{x} - \bar{x}\bvec{1}
\]

\smallskip

We can think of $\bvec{e}_x$ as a ``mean centered'' version of $\bvec{x}$ , i.e. it's the vector we get when we subtract the mean of $\bvec{x}$, $\bar{x}$, from every element of $\bvec{x}$.

\medskip

Important relationships for mean-centered vectors:

\begin{itemize}
  \item The variance of $X$ is proportional to  $|\bvec{e}_x|^2$  
  \item The standard deviation of $X$ is proportional to  $|\bvec{e}_x|$
\end{itemize}

\medskip

{\small
For convenience, I will sometimes state the variables of interest are mean centered and use the notation $\bvec{x}$ instead of $\bvec{e}_x$ so as to avoid a proliferation of subscripts.
}


\end{frame}
%===========================================================



%===========================================================
\begin{frame}
  \frametitle{Covariance and correlation in vector geometric terms}

Let $X$ and $Y$ be variables of interest, and let $\bvec{x}$ and $\bvec{y}$ be their corresponding mean centered vector representations.

\begin{center}

\begin{tikzpicture}[x=0.5cm, y=0.5cm]

\draw[thick,->] (0,0) -- (3,2);
\draw (3,2) node[right] {$\bvec{x}$};

\draw[thick,->] (0,0) -- (5,-1);
\draw (5,-1) node[above] {$\bvec{y}$};

\draw (0,0) +(-11:0.5cm) arc (-11:34:0.5cm);
\path (0,0) ++(10:0.7cm) node[font=\footnotesize] {$\theta$};

\end{tikzpicture}
\end{center}

Vector formulas for covariance and correlation:
\begin{eqnarray*}
\text{Covariance: } \cov(X,Y) &=&  \frac{\bvec{x} \cdot \bvec{y}}{n-1}\\
\text{Correlation: } \corr(X,Y) &=&  \frac{\bvec{x} \cdot \bvec{y}}{|\bvec{x}||\bvec{y}|} = \cos \theta
\end{eqnarray*}

\begin{alertblock}{Geometric interpretation of correlation}
    The correlation between two variables $X$ and $Y$ is equivalent to the cosine of the angle between their mean-centered vector representations!
\end{alertblock}
    

\end{frame}
%===========================================================




\end{document}
