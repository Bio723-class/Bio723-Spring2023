%\useoutertheme[glossy]{wuerzburg}
\useinnertheme[shadow,outline]{chamfered}
%\usecolortheme{shark}
\usecolortheme{beaver}
\beamertemplatenavigationsymbolsempty


\usepackage{fontspec}      % support opentype text fonts
\usepackage{amsmath}
\usepackage[mathrm=sym,mathbf=sym]{unicode-math}
                           % support opentype math fonts, including bold

% Slightly reduce font size.
\defaultfontfeatures{Scale=.85}
%
% The explicit .otf extension tells the engines to find the font via
% filenames in the TeX trees, rather than as system-installed fonts.
%
\setmainfont[% main rm
  Ligatures=TeX,
      ItalicFont=LucidaBrightOT-Italic.otf,
        BoldFont=LucidaBrightOT-Demi.otf,
  BoldItalicFont=LucidaBrightOT-DemiItalic.otf,
  ]{LucidaBrightOT.otf}
%
\setsansfont[% main sans
  Ligatures=TeX,
      ItalicFont=LucidaSansOT-Italic.otf,
        BoldFont=LucidaSansOT-Demi.otf,
  BoldItalicFont=LucidaSansOT-DemiItalic.otf,
  ]{LucidaSansOT.otf}
%
\setmonofont[% main typewriter
      ItalicFont=LucidaSansTypewriterOT-Oblique.otf,
        BoldFont=LucidaSansTypewriterOT-Bold.otf,
  BoldItalicFont=LucidaSansTypewriterOT-BoldOblique.otf,
  ]{LucidaSansTypewriterOT.otf}
%

\setmathfont{LucidaBrightMathOT.otf}[BoldFont=LucidaBrightMathOT-Demi]
%
% The specialized one-off fonts:
\newfontface\LucidaBlackletter{LucidaBlackletterOT.otf}
\newfontface\LucidaCalligraphy{LucidaCalligraphyOT.otf}
\newfontface\LucidaHandwriting{LucidaHandwritingOT.otf}
%
% GrandeMono and Console fonts for an example:
\newfontface\LucidaGrandeMono{LucidaGrandeMonoDK.otf}
\newfontface\LucidaConsole{LucidaConsoleDK.otf}

\usepackage{fancyvrb}

%% Fancy syntax coloring via pygments
%\usepackage{minted}
%\definecolor{bg}{rgb}{0.95,0.95,0.95}
%\usemintedstyle{borland}


% \newenvironment{Rcode}
% {\VerbatimEnvironment
%  \begin{minted}[fontsize=\scriptsize,baselinestretch=1]{r}}%
% {\end{minted}}

% \newenvironment{Pcode}
% {\VerbatimEnvironment
% \begin{minted}[fontsize=\scriptsize,baselinestretch=1]{python}}%
% {\end{minted}}

% \newenvironment{Code}[1]
% {\VerbatimEnvironment
 % \begin{minted}[fontsize=\scriptsize,baselinestretch=1]{#1}}%
% {\end{minted}}


\usepackage{textfit} % commands \scaletoheight{height}{text} and \scaletowidth{width}{text}

\usepackage{tikz}

\usepackage{tcolorbox}

\usepackage{subcaption}

\newtheorem{Alert}{Alert}
\newtheorem{Highlight}{Highlight}

\newcommand{\Species}[1]{{\rmfamily \itshape #1}}
\newcommand{\Real}{\ensuremath{\mathbb{R}}}
\newcommand{\RealN}{\ensuremath{\mathbb{R}^n}}
\newcommand{\RealP}{\ensuremath{\mathbb{R}^p}}
\newcommand{\Mtx}[1]{\ensuremath{\bm{#1}}}
\newcommand{\Inv}[1]{\ensuremath{#1^{-1}}}
\newcommand{\InvMtx}[1]{\ensuremath{\bm{#1}^{-1}}}
\newcommand{\Red}[1]{\textcolor{red}{#1}}
\newcommand{\PsInv}[1]{\ensuremath{\bm{#1}^{+}}}
\DeclareMathOperator{\cov}{cov}
\DeclareMathOperator{\corr}{corr}

\usepackage{booktabs}


\usepackage{pifont}
\newcommand{\weblink}{\ding{43}}  % hand with pointing finger

\definecolor{links}{HTML}{2A1B81}
\hypersetup{colorlinks,linkcolor=,urlcolor=magenta}




% --- Override \vec with an invocation of \xvec.
\let\stdvec\vec
\newcommand{\bvec}[1]{\vec{\mathbf{#1}}}

